\sffamily
\chapter{Introdução} 

O que [\ldots] acarretaria problemas com a identidade (Júlio César). Com base nestas considerações, além do fato de que, em 1882, Frege provara as leis básicas da aritmética (carta a Anton Marty), parece-nos, \ac{iot}.

\section{Seção Secundária1} 

dsdsds


\section[Seção Secundária2]{Seção Secundária2} 

plausível que estas provas foram executadas adicionando-se o \textbf{PH} ao sistema lógico de Begriffsschrift. Mostramos que, nas provas dos axiomas de Peano a partir de \textbf{PH} dentro da conceitografia, nenhum uso é feito de (\textbf{BB}). Destarte, não é necessária a introdução.

\subsection{Seção Terciária}

\begin{alineas}
	\item linha 1:
	\begin{alineas}
		\item subalinea 1;
		\item subalinea 2;
	\end{alineas}
	\item linha 2:
	\begin{subalineas}
		\item subalinea 1;
		\item subalinea 2;
	\end{subalineas}
	\item linha 3:
	\begin{incisos}
		\item subalinea 1;
		\item subalinea 2;
	\end{incisos}
	\item linha 4.
\end{alineas}


\begin{table}[htb]
	\IBGEtab{%
		\caption{Um Exemplo de tabela alinhada que pode ser longa ou curta,
		conforme padrão IBGE.}%
		\label{tabela-ibge}
		}{%
		\begin{tabular}{ccc}
			\toprule
			Nome           & Nascimento & Documento      \\
			\midrule \midrule
			Maria da Silva & 11/11/1111 & 111.111.111-11 \\
			\bottomrule
		\end{tabular}%
		}{%
		\fonte{Produzido pelos autores}%
		\nota{Esta é uma nota, que diz que os dados são baseados na
		regressão linear.}%
		\nota[Anotações]{Uma anotação adicional, seguida de várias outras.}%
	}
\end{table}

\begin{table}[ht]
	\ABNTEXfontereduzida
	\caption[Níveis de investigação]{Níveis de investigação.}
	\label{tab-nivinv}
	\begin{tabular}{p{2.6cm}|p{6.0cm}|p{2.25cm}|p{3.40cm}}
		%\hline
		\textbf{Nível de In\-ves\-ti\-ga\-ção} & \textbf{Insumos}                                                      & \textbf{Sis\-te\-mas de In\-ves\-ti\-ga\-ção} & \textbf{Produtos}      \\
		\hline
		Meta-nível                               & Filosofia\index{filosofia} da Ciência                                & Epistemologia                                   &                        
		Paradigma  \\
		\hline
		Nível do objeto & Paradigmas do metanível e evidências do nível inferior &
		Ciência  & Teorias e modelos \\
		\hline
		Nível inferior                           & Modelos e métodos do nível do objeto e problemas do nível inferior & Prática                                        & Solução de problemas \\
		% \hline
	\end{tabular}
	%\legend{Fonte: \citeonline{santiago2014google}}
\end{table}

%https://tex.stackexchange.com/questions/138/what-are-underfull-hboxes-and-vboxes-and-how-can-i-get-rid-of-them

