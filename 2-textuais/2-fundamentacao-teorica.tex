\chapter{\bfseries Fundamentaçao Teórica}


    % \begin{tabular}{lll}               
    %     \verb|\ac{iot}|        & first use                   & \ac{iot}      \\
    %     \verb|\ac{iot}|        & second use                  & \ac{iot}      \\
    %     \verb|\acl{iot}|       & force the long version      & \acl{iot}     \\
    %     \verb|\acs{iot}|       & force the short version     & \acs{iot}     \\
    %     \verb|\acf{iot}|       & force the full version      & \acf{iot}     \\
    %     \verb|\aclp{iot}|      & force the plural version    & \aclp{iot}    \\ %\acp, \acsp, \acfp
    %     \verb|\acfi{iot}|      & force the italic version    & \acfi{iot}    \\
    %     \verb|\ac*{iso}|       & first use, don't mark used  & \ac*{iso}     \\ % same with all other commands
    %     \verb|\ac{iso}|        & second use                  & \ac{iso}      \\
    %     \verb|\aclp{iso}|      & force the plural version    & \aclp{iso}    \\
    %     \verb|\acused{ICANN}|  & mark as used, no printing   & \acused{icann}\\
    %     \verb|\ac{ICANN}|      & second use, as if it is 1st & \ac{icann}    \\
    %     \verb|\ac{gpio}|       & acronym used, but not listed& \ac{gpio}     \\  
    %     \verb|\ac{gpio}|       & second use                  & \ac{gpio}    
    % \end{tabular}    
    


\nomenclature{$s$}{O semi-perímetro do triângulo (metade do perímetro)}

%\nomenclature[prefix]{symbol}{description}

\begin{equation}a=\frac{N}{A}\end{equation}%
\nomenclature{$a$}{The number of angels per unit area}%
\nomenclature{$N$}{The number of angels per needle point}%
\nomenclature{$A$}{The area of the needle point}%

The equation $\sigma = m a$%
\nomenclature{$\sigma$}{The total mass of angels per unitarea}%
\nomenclature{$m$}{The mass of one angel}follows easily.

follows easily.



The equation $\sigma = m a$%
\nomenclature{$\sigma$}{The total mass of angels per unitarea}%




Ilustrações ABNT NBR 14724:2011:

Qualquer que seja o tipo de ilustração, sua identificação aparece na parte superior, precedida da palavra designativa (desenho, esquema, fluxograma, fotografia, gráfico, mapa, organograma, planta, quadro, retrato, figura, imagem, entre outros), seguida de seu número de ordem de ocorrência no texto, em algarismos arábicos, travessão e do respectivo título. 

Após a ilustração, na parte inferior, indicar a fonte consultada (elemento obrigatório, mesmo que seja produção do próprio autor), legenda, notas e outras informações necessárias à sua compreensão (se houver). A ilustração deve ser citada no texto e inserida o mais próximo possível do trecho a que se refere.