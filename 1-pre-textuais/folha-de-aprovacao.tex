% Isto é um exemplo de Folha de aprovação, elemento obrigatório da NBR
% 14724/2011 (seção 4.2.1.3). Você pode utilizar este modelo até a aprovação
% do trabalho. Após isso, substitua todo o conteúdo deste arquivo por uma
% imagem da página assinada pela banca com o comando abaixo:
%
% \includepdf{folhadeaprovacao_final.pdf}
%

%ajusta tamanho linha textual horizontal se necessário
\setlength{\ABNTEXsignwidth}{12cm} 

%coloque 1pt se desejar linha. Comumente, linha de assinatura é utilizada para pessoas não letradas.
\setlength{\ABNTEXsignthickness}{0pt} 

%espaçamento entre assinaturas
\setlength{\ABNTEXsignskip}{0.7cm} 


\begin{folhadeaprovacao}
	\begin{center} 
		{\ABNTEXchapterfont\imprimirautor}

        \vspace*{\fill}\vspace*{\fill}
        \begin{center}
			\ABNTEXchapterfont\bfseries\imprimirtitulo
        \end{center}
        \vspace*{\fill}
		% Se desejar preâmbulo, descomente as linhas abaixo
		% \hspace{.45\textwidth} 
		% \begin{minipage}{.5\textwidth}
		% 	\imprimirpreambulo
		% \end{minipage}
		\vspace*{\fill}
    \end{center}
    
    \begin{description}[align=right,labelwidth=4cm]
    \item [Status] Trabalho aprovado.
    \item [Local e data de defesa] \imprimirlocal, \defesadatacompletacomdiadasemana.
    \end{description}

%\the\year.

	\assinatura{\imprimirorientador\\{\footnotesize \proforientadortitulacao\\ (Orientador UTFPR)}}
	\assinatura{\imprimircoorientador\\{\footnotesize \profcoorientadortitulacao\\(Co-orientador UTFPR)}} 
    
    \assinatura{\profpresidentebanca\\{\footnotesize \profpresidentebancatitulacao\\(Presidente da Banca UTFPR)}} 
    \assinatura{\profbancaA\\{\footnotesize \profbancaAtitulacao\\(Membro1 Banca UTFPR)}}
    \assinatura{\profbancaB\\{\footnotesize \profbancaBtitulacao\\(Membro2 Banca UTFPR)}}


	\vspace*{\fill}
	\begin{center}
	\noindent Folha de Aprovação assinada encontra-se arquivada na Coordenação do Curso.
	\end{center} 
\end{folhadeaprovacao} 