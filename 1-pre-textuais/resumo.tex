%NBR 6028: de 150 a 500 palavras os de trabalhos acadêmicos (teses, dissertações e outros) e relatórios técnico-cientifícos;

\begin{resumo}[\protect\bfseries Resumo]  
    Nos \textit{Fundamentos da Aritmética} (§68), Frege propõe definir explicitamente o operador-abstração `o número de...' por meio de extensões e, a partir desta definição, provar o Princípio de Hume (\textbf{PH}). Contudo, a prova imaginada por Frege depende de uma fórmula (\textbf{BB}) não derivável no sistema em 1884. Acreditamos que a distinção entre sentido e referência e a introdução dos valores de verdade como objetos foram motivadas para justificar a introdução do Axioma IV, a partir do qual um análogo de (\textbf{BB}) é provável. Com (\textbf{BB}) no sistema, a prova do Princípio de Hume estaria garantida. Concomitantemente, percebemos que uma teoria unificada das extensões só é possível com a distinção entre sentido e referência e a introdução dos valores de verdade como objetos. Caso contrário, Frege teria sido obrigado a introduzir uma série de \textbf{Axiomas V} no seu sistema, o que acarretaria problemas com a identidade (Júlio César). Com base nestas considerações, além do fato de que, em 1882, Frege provara as leis básicas da aritmética (carta a Anton Marty), parece-nos perfeitamente plausível que estas provas foram executadas adicionando-se o \textbf{PH} ao sistema lógico de Begriffsschrift. Mostramos que, nas provas dos axiomas de Peano a partir de \textbf{PH} dentro da conceitografia, nenhum uso é feito de (\textbf{BB}). Destarte, não é necessária a introdução do Axioma IV no sistema e, por conseguinte, não são necessárias a distinção entre sentido e referência e a introdução dos valores de verdade como objetos. Disto, podemos concluir que, provavelmente, a introdução das extensões nos \textit{Fundamentos} foi um ato tardio; e que Frege não possuía uma prova formal de \textbf{PH} a partir da sua definição explícita. Estes fatos também explicam a demora na publicação das \textit{Leis Básicas da Aritmética} e o descarte de um manuscrito quase pronto (provavelmente, o livro mencionado na carta a Marty). 
\vspace{\onelineskip} 

\noindent \textbf{Palavras-chave}: \palavraschaves
\end{resumo}