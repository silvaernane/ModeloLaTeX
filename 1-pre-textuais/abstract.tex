\begin{resumo}[\protect\bfseries Abstract] 
    \begin{otherlanguage*}{english}
    In \textit{The Foundations of Arithmetic} (§68), Frege proposes to define explicitly the abstraction operator `the number of...' by means of extensions and, from this definition, to prove Hume's Principle (\textbf{HP}). Nevertheless, the proof imagined by Frege depends on a formula (\textbf{BB}), which is not provable in the system in 1884. We believe that the distinction between sense and reference as well as the introduction of Truth-Values as objects were motivated in order to justify the introduction of Axiom IV, from which an analogous of (\textbf{BB}) is provable. With (\textbf{BB}) in the system, the proof of HP would be guaranteed. At the same time, we realize that a unified theory of extensions is only possible with the distinction between sense and reference and the introduction of Truth-Values as objects. Otherwise, Frege would have been obliged to introduce a series of \textbf{Axioms V} in his system, what cause problems regarding the identity (Julius Caesar). Based on these considerations, besides the fact that in 1882 Frege had proved the basic laws of Arithmetic (letter to Anton Marty), it seems perfectly plausible that these proofs carried out by adding \textbf{HP} to the Begriffsschrift's logical system. We show that in the proofs of Peano's axioms from \textbf{HP} within the begriffsschrift, (\textbf{BB}) is not used at all. Thus, the introduction of Axiom IV in the system is not necessary and, consequently, neither the distinction between sense and reference nor the introduction of Truth-Values as objects. From these findings we may conclude that probably the introduction of extensions in The \textit{Foundations} was a late act; and that Frege did not hold a formal proof of \textbf{HP} from his explicit definition. These facts also explain the delay in the publication of \textit{the Basic Laws of Arithmetic} and the abandon of a manuscript almost finished (probably the book mentioned in the letter to Marty).
    \vspace{\onelineskip} 
    
    \noindent \textbf{Keywords}: \palavraschavesIngles 
    
    \end{otherlanguage*} 
    \end{resumo} 