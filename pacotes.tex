% ---
% Não tente entender este .tex
% Aqui tem ajustes hardcore sombreando ABNTEX2 para finos ajustes de ABNT.
% ---

% ---
% PDF/A
% ---
%
% PDF/A é um padrão ISO destinado ao arquivamento a longo prazo de documentos eletrônicos. Enfatiza a autonomia e a reprodutibilidade, bem como os metadados legíveis por máquina.
%
% A pedido da UTFPR-FB-Biblioteca, valide seu PDF/A em
% https://www.pdf-online.com/osa/validate.aspx
%
% PDF gerado estará ok se a mensagem for: "The document does conform to the PDF/A-3b standard."


% ---
% Pacotes básicos 
% ---

\usepackage[T1]{fontenc}		% Selecao de codigos de fonte.
\usepackage[utf8]{inputenc}		% Codificacao do documento -conversão automática dos acentos
\usepackage{ae,aecompl}
%http://dsanta.users.ch/resources/type1.html

\usepackage{lmodern}			% Usa a fonte Latin Modern			
%\usepackage{lastpage}			% Usado pela Ficha catalográfica
\usepackage{indentfirst}		% Indenta o primeiro parágrafo de cada seção.
\usepackage{color}				% Controle das cores
\usepackage{graphicx}			% Inclusão de gráficos
\usepackage{microtype} 			% para melhorias de justificação

\usepackage{setspace} % double spacing
\usepackage{url}
\usepackage[brazil]{babel}
\usepackage{helvet}
\usepackage[top=3cm, bottom=2cm, left=3cm, right=2cm]{geometry}
\usepackage{hyperref}


% % ---
\hbadness=99999 %That's just TeX alerting you that it was unable to typeset your document perfectly. The \hbadness variable doesn't affect the typography at all; it just tells TeX the threshold for printing its annoying "Underfull \hbox (badness xxxx) in paragraph..." warnings. 

% ---
% Pacotes de citações
% ---
% \usepackage[brazilian,hyperpageref]{backref}	 % Paginas com as citações na bibl
\usepackage[brazilian]{backref}

%Sugestão
\usepackage[alf,bibjustif,
    abnt-emphasize=bf,
    bibjustif,
    recuo=0cm,
    abnt-url-package=url,
    abnt-refinfo=yes,
    abnt-repeated-title-omit=yes,
    abnt-full-initials=yes, %(yes) nome por extenso, (no) apenas iniciais
    abnt-etal-list=3,
    abnt-etal-cite=3,  %abreviar com mais de 3 autores
    abnt-nbr10520=2002,
    abnt-thesis-year=final
]{abntex2cite}

% --- 
% Espaçamentos entre linhas e parágrafos 
% --- 
%
% O tamanho do parágrafo é dado por:
\setlength{\parindent}{1.3cm}
%
% Controle do espaçamento entre um parágrafo e outro:
\setlength{\parskip}{0.2cm}  % tente também \onelineskip

% acronyms e glossarios
%\usepackage{glossaries} e pacote acro dão erro no abntex2
\usepackage[printonlyused,withpage]{acronym}

\hbadness=99999

% --- 
% CONFIGURAÇÕES DE PACOTES
% --- 

% ---
% Pacote de Formatação de URL 
% ---
\usepackage{url} %url clicáveis
\makeatletter  \def\url@leostyle{%
    \@ifundefined{selectfont}{\def\UrlFont{\sf}}{\def\UrlFont{\small\ttfamily}}}
\makeatother \urlstyle{leo}
\urlstyle{same} %url sem fonte diferente


\usepackage{paralist} %https://latex.org/forum/viewtopic.php?t=3376


% ---
% Pacote Gráfico 
% ---
\usepackage{graphicx} %graphbox: Extend graphicx to improve placement of graphics 
\usepackage{subcaption} %Multiple images / subfigures in LaTeX
\ifpdf
    \DeclareGraphicsExtensions{%
        .png,.PNG,%
        .pdf,.PDF,%
        .jpg,.mps,.jpeg,.jbig2,.jb2,.JPG,.JPEG,.JBIG2,.JB2}
\else
    \DeclareGraphicsExtensions{.eps}
\fi
\graphicspath{%
    {2-textuais/figs/}%
        {3-pos-textuais/figs-anexo/}%
        {3-pos-textuais/figs-apendice/}%
}
%\graphicspath{{subdir1/}{subdir2/}{subdir3/}...{subdirn/}}


% Adicionar mais de um tipo de entrada à Lista de ilustrações: Mapa e Desenho
%% Mapa
\newcommand{\mapaname}{Mapa}
\newfloat[chapter]{mapa}{lof}{\mapaname}
\newlistentry{mapa}{lof}{0}
\counterwithout{mapa}{chapter}
\renewcommand{\cftmapaname}{\mapaname\space}
\renewcommand*{\cftmapaaftersnum}{\hfill--\hfill}
\setfloatlocations{mapa}{hbtp} % configurando posicionamento padrão

%% Desenho
\newcommand{\desenhoname}{Desenho}
\newfloat[chapter]{desenho}{lof}{\desenhoname}
\newlistentry{desenho}{lof}{0}
\counterwithout{desenho}{chapter}
\renewcommand{\cftdesenhoname}{\desenhoname\space}
\renewcommand*{\cftdesenhoaftersnum}{\hfill--\hfill}
\setfloatlocations{desenho}{hbtp} % configurando posicionamento padrão

% % ---
% % Configurações do pacote backref
% % Usado sem a opção hyperpageref de backref
% \renewcommand{\backrefpagesname}{Citado na(s) página(s):~}
% % Texto padrão antes do número das páginas
% \renewcommand{\backref}{}
% % Define os textos da citação
% \renewcommand*{\backrefalt}[4]{
% 	\ifcase #1 %
% 		Nenhuma citação no texto.%
% 	\or
% 		Citado na página #2.%
% 	\else
% 		Citado #1 vezes nas páginas #2.%
% 	\fi}%
% % ---

% ---
% Arrumando ABNTEX2: em tam. de fonte de Titulo1, mas estah encadeado/herdado
\renewcommand{\ABNTEXpartfontsize}{\normalsize}
\renewcommand{\ABNTEXchapterfontsize}{\normalsize} %diminuindo tam de fonte 
\renewcommand{\ABNTEXsectionfontsize}{\normalsize}
\renewcommand{\ABNTEXsubsectionfontsize}{\normalsize}
\renewcommand{\ABNTEXsubsubsectionfontsize}{\normalsize}

\renewcommand{\cftpartfont}{\bfseries\normalsize} %diminiu a fonte de Apendices e Anexos
\renewcommand{\cftpartpagefont}{\bfseries\normalsize}%diminiu o num de pag da fonte de Apendices e Anexos

%----------------------------------------------------
% pacotes para insercao de codigo fonte
% ---

\usepackage{inconsolata}
\definecolor{pblue}{rgb}{0.13,0.13,1}
\definecolor{pgreen}{rgb}{0,0.5,0}
\definecolor{pred}{rgb}{0.9,0,0}
\definecolor{pgrey}{rgb}{0.46,0.45,0.48}

\definecolor{javared}{rgb}{0.6,0,0} % for strings
\definecolor{javagreen}{rgb}{0.25,0.5,0.35} % comments
\definecolor{javapurple}{rgb}{0.5,0,0.35} % keywords
\definecolor{javadocblue}{rgb}{0.25,0.35,0.75} % javadoc

\definecolor{lightpurple}{rgb}{0.8,0.8,1}

%--------------

\usepackage{listings}
\lstset{
    language=Java,
    basicstyle=\ttfamily,
    breakatwhitespace=true,
    breaklines=true,
    % commentstyle=\color{javagreen},
    % keywordstyle=\color{javapurple}\bfseries,
    % morecomment=[s][\color{javadocblue}]{/**}{*/}, %added
    numbers=left, %added
    numbersep=10pt, %added
    % numberstyle=\tiny\color{black}, %added
    showspaces=false,
    showstringspaces=false
    stepnumber=2, %added
    % stringstyle=\color{javared},
    tabsize=2, %added
}

% Para LaTeX pode-se usar
% https://texblog.org/2011/06/11/latex-syntax-highlighting-examples/

%---------------------------------------------

% Novo list Para Sumário de Codigos

\newcommand{\codigoname}{Código}
\newcommand{\listofcodigosname}{\bfseries LISTA DE EXCERTOS DE CÓDIGO-FONTE}

\newfloat[chapter]{codigo}{loc}{\codigoname}
\newlistof{listofcodigos}{loc}{\listofcodigosname}
\newlistentry{codigo}{loc}{0}

% configurações para atender às regras da ABNT
\setfloatadjustment{codigo}{\centering}
\counterwithout{codigo}{chapter}
\renewcommand{\cftcodigoname}{\codigoname\space}
\renewcommand*{\cftcodigoaftersnum}{\hfill--\hfill}

% Configuração de posicionamento padrão:
\setfloatlocations{codigo}{hbtp}

% ===============================

% Lista de Símbolos
\usepackage{nomencl}
\makenomenclature %uma instrução para ser elaborada a lista

% ----------------------------------------------------
% Tabelas
% ---
\usepackage{longtable}
\usepackage{array}
\newcolumntype{L}[1]{>{\raggedright\let\newline\\\arraybackslash\hspace{0pt}}m{#1}}
\newcolumntype{C}[1]{>{\centering\let\newline\\\arraybackslash\hspace{0pt}}m{#1}}
\newcolumntype{R}[1]{>{\raggedleft\let\newline\\\arraybackslash\hspace{0pt}}m{#1}}

% ----------------------------------------------------
% Todo e revisao de textos
% ---

\usepackage{todonotes}
\usepackage[normalem]{ulem}
\setlength{\marginparwidth}{2cm}
%http://www.ufpa.br/heliton/arquivos/aplicativos/latex/minicurso_latex_2011.pdf


% ----------------------------------------------------
% Quadros
% ---

% \usepackage{trivfloat}
% \trivfloat{quadro}
% %https://bibliotecafea.com/2012/09/21/tabela-e-quadro-diferencas/

% \renewcommand{\listquadroname}{Lista de Quadros}

% Novo list of (listings) para QUADROS

\newcommand{\quadroname}{Quadro}
\newcommand{\listofquadrosname}{\bfseries Lista de quadros}

\newfloat[chapter]{quadro}{loq}{\quadroname}
\newlistof{listofquadros}{loq}{\listofquadrosname}
\newlistentry{quadro}{loq}{0}

% configurações para atender às regras da ABNT
\counterwithout{quadro}{chapter}
\renewcommand{\cftquadroname}{\quadroname\space}
\renewcommand*{\cftquadroaftersnum}{\hfill--\hfill}

% Configuração de posicionamento padrão:
\setfloatlocations{quadro}{hbtp}

%....e

%Inserindo Quadros com o pacote longtable, https://github.com/abntex/abntex2/wiki/HowToCriarNovoAmbienteListing
\usepackage{longtable,ltcaption} % para as tabelas

%--------------------

\usepackage{caption}

\usepackage[margin=10pt,font=normalsize,labelfont=bf,labelsep=endash]{caption}


\addto\captionsbrazil{%
    \renewcommand\listfigurename{\bfseries Lista de Ilustrações}
    \renewcommand\listtablename{\bfseries Lista de tabelas}
    \renewcommand\contentsname{\bfseries Sumário}
    \renewcommand{\bibname}{\bfseries Refer\^encias}
    \renewcommand{\indexname}{\uppercase{\bfseries \'Indice}}
}

\renewcommand{\familydefault}{\sfdefault} %suprime fonte em abntex e força nova fonte em todo o doc.

% Finalmente, corrigi negrito nos numeros em chap e sect.
\renewcommand{\chapnumfont}{\bfseries\memRTLraggedright} %ok
\setsecheadstyle{\bfseries\memRTLraggedright\uppercase} % Set \section style

% Finalmente, corrigi Upercase no TOC/DOC
% TOC Spacing in Memoir
% https://tex.stackexchange.com/questions/60317/toc-spacing-in-memoir
%\setlength{\cftbeforechapterskip}{10pt}

\makeatletter
\renewcommand*{\l@section}[2]
{%
    \l@chapapp{\uppercase{#1}}{#2}{\cftsectionname}
}
\makeatother

\renewcommand{\orientadorname}{\normalsize Orientador:}
\renewcommand{\coorientadorname}{\normalsize Coorientador:}

\renewcommand{\apendicename}{\normalsize\bfseries AP\^ENDICE}
\renewcommand{\apendicesname}{\bfseries AP\^ENDICES}
\renewcommand{\anexoname}{\normalsize\bfseries ANEXO}
\renewcommand{\anexosname}{\bfseries ANEXOS}



